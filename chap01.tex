\chapter{Basic Definitions}

\section{Information Theory}

The basic intuition behind information theory is that learning that an unlikely
event has occurred is more informative than learning that a likely event has
occurred \citep{Goodfellow-et-al-2016}. We define \newterm{self-information} of
a random event $x$ as

% \[
%    𝐉 = ∇×𝐇 \qquad 𝐁 = μ₀(𝐌 + 𝐇)
% \]

\begin{equation}
  I(x) = -\log P(x).
\end{equation}

The overall information in a probability distribution is quantified by
\newterm{Shannon entropy}:

\begin{equation}
  H(X) = \E_{X \sim P} [ I(X) ] = -\E_{X \sim P} [ \log P(x) ]
\end{equation}

We can extend this to the case of a continuous random variable with a
probability density function $f$, in which case we call $H(f)$ the
\newterm{differential entropy} and define it as

\begin{equation}
  H(f) = - \int_X f(x) \log f(x) \dx
\end{equation}

\section{Calculus of Variations}

Before we can derive the Gaussian distribution, we first introduce a few
concepts from the calculus of variations.

A \newterm{functional} is a mapping which assigns real numbers to each function
belonging to some class \citep{gelfand2012calculus}. We can say that a
functional is a special kind of function, where the argument is itself a
function. In our case the functional will be the differential entropy $H(f)$.
Similarly to derivatives and partial derivatives in real analysis, we can take
\newterm{functional derivatives} of a functional $H(f)$ with respect

As full derivation of functional derivatives is outside the scope of this
thesis, we only present the following theorem without proof, and refer the
reader to the above cited textbook for a proof \todo{find the proof}.

For differentiable functions $f(\vx)$ and $g(y, \vx)$ with continuous
derivatives, the following holds

\begin{equation}
  \frac{\delta}{\delta f(\vx)} \int g(f(\vx), \vx) d \vx = \frac{\partial}{\partial y} g(f(\vx), \vx).
\end{equation}

\todo{rewrite, copied over from DLB} To gain some intuition for this identity,
one can think of $f(\vx)$ as being a vector with uncountably many elements,
indexed by a real vector $\vx$. In this (somewhat incomplete) view, the
identity providing the functional derivatives is the same as what we would
obtain for a vector $\vtheta \in \sR^n$ indexed by positive integers:

\begin{equation}
  \frac{\partial}{\partial \theta_i} \sum_j g(\theta_j, j) = \frac{\partial}{\partial \theta_i} g(\theta_i, i).
\end{equation}

\section{Maximum Entropy Probability Distribution}

In this section we derive the Gaussian distribution as a distribution which
maximizes differential entropy when the mean and variance are fixed. Similarly
as we would optimize a function by computing its gradient, setting it to zero
and solving the equation, we can optimize a functional by solving for a
function where the variational derivative is zero at every point.

Because we want our resulting functional to be a probability distribution, we
need its density to integrate to $1$, specifically

\begin{equation}
  \int f(x) \dx = 1
\end{equation}

We also want the mean and variance to be fixed constants $\mu$ and $\sigma^2$
respectively. This gives us the following

\begin{align}
  \int x f(x) \dx = \mu \\
  \int (x - \mu) ^2 f(x) \dx = \sigma^2
\end{align}

We can solve for all of these constraints simultaneously using Lagrange
multipliers. The Lagrangian has the following form:

\begin{align}
  \begin{split}
    \gL[f] ={}& H[f] + \lambda_1 \left( \int f(x) \dx - 1 \right) + \lambda_2 \left( \int x f(x) \dx - \mu \right) + \\
              &+ \lambda_3 \left( \int (x - \mu)^2 f(x) \dx - \sigma^2 \right)
  \end{split} \\
  %
  \begin{split}
    ={}& -\int f(x) \log f(x) \dx  + \lambda_1 \left( \int f(x) \dx - 1 \right) + \\
       &+ \lambda_2 \left( \int x p(x) \dx - \mu \right) + \lambda_3 \left( \int (x - \mu)^2  f(x) \dx - \sigma^2 \right)
  \end{split} \\
  %
  \begin{split}
    ={}& \int ( - f(x) \log f(x) + \lambda_1 f(x) - \lambda_1 + \\
       &+ \lambda_2 x p(x) - \lambda_2 \mu + \lambda_3 (x - \mu)^2  f(x) - \lambda_3 \sigma^2 ) \dx \\
  \end{split} \\
  %
  \begin{split}
    ={}& \int ( - f(x)\log f(x) + \lambda_1 f(x) + \lambda_2 x f(x) + \lambda_3 (x - \mu)^2  f(x) ) \dx + \\
       &- \lambda_1 - \lambda_2 \mu - \lambda_3 \sigma^2 \label{eq:gauss-lagrangian}
  \end{split}
\end{align}

\todo{check the functional derivative, so far only copied from DLB, fix
alignment of = in 1.11} We set the following functional derivative of
\eqref{eq:gauss-lagrangian} equal to $0$:

\begin{equation}
  \forall x, \frac{\delta}{\delta f(x)} \gL = \lambda_1 + \lambda_2 x + \lambda_3 (x - \mu)^2 - 1 - \log f(x) = 0.
\end{equation}

Now simply solving for $f(x)$ we obtain:

\begin{equation}
  f(x) = \exp \left( \lambda_1 + \lambda_2 x + \lambda_3 (x - \mu)^2 - 1 \right).
\end{equation}

\todo{rewrite, mostly written as in DLB}

To solve the optimization problem we can choose any value for the $\lambda$. We
set $\lambda_1 = 1 - \log \sigma \sqrt{2 \pi}$, $\lambda_2 = 0$, and $\lambda_3
= - \frac{1}{2 \sigma^2}$ to obtain the density of a Gaussian distribution

\begin{equation}
  f(x) = \gN(x | \mu, \sigma^2).
\end{equation}

This derivation is one of the reason why a Gaussian distribution is so useful.
\todo{rewrite, copied from DLB} Because the normal distribution has the maximum
entropy, we impose the least possible amount of structure by making this
assumption.

