A machine learning algorithm is an algorithm that can learn from data. A
commonly used definition is: "A computer program is said to learn form
experience $E$ with respect to some class of tasks $T$ and performance measure
$P$, if its performance at tasks in $T$, as measured by $P$, improves with
experience $E$ \citep{mitchell-mldef}. We only provide this definition for
completeness and our following definitions will assume an intuitive translation
to this definition.

Machine learning models are traditionally split into two categories, parametric
and non-parametric. Parametric models have a fixed number of learnable parameters
which are trained on a subset of the data called the training set. On the other hand,
non-parametric models typically have a variable number of parameters which depends on
the size of the data. This could mean that there is a parameter associated with each
data point. Non-parametric does not mean that there are no parameters.

Most machine learning algorithms also have hyperparameters, which are
parameters that the learning algorithm can not learn itself, and they usually
control the its behavior. A parameter could be chosen to be a hyperparameter
because it would not be reasonable to infer its value from the training data.
An example could be the type of optimizer being used to train a neural network
(e.g. SGD or Adam \citep{kingma2014adam}). The machine learning practicioner
would not consider the optimizer to be a property of the data distribution, and
as a reasult treat it as a hyperparameter set externally, rather than trying to
infer it.

It could also be set as a hyperparameter simply for practical reasons, where in
theory the parameter could be learned from the data, but optimizing it directly
would be too difficult. An example here could be the learning rate of a
stochastic gradient descent optimizer. Even automatic differentiation
\citep{maclaurin2015autograd} allows us to compute the gradient flow through
arbitrary code, in practice it is not being used to compute the gradients of
hyperparameters like the parameters of an optimizer, e.g.\ the learning rate.
Instead, the learning rate is treated as a hyperparameter and is set ahead of
time, or according to a fixed schedule. Even when early stopping or other
heuristic methods are employed, it would still be treated as a hyperparameter
from the perspective of the learning algorithm.

Bayesian statistics allows us to take a principled approach to setting
hyperparameters. We would simply set a prior distribution on each
hyperparameter and marginalize over them, but unfortunately this has two
problems. The prior distribution almost always has a parameter of its own, such
as the rate of a poisson distribution. We could go one step further and specify
a prior on these parameters, which are often called hyperpriors.  But the real
problem of bayesian methods is that the integral in the marginalization often
ends up being intractable, and in the case of more complicated as neural
networks, we are already computatationally bottlenecked and can not use methods
such as Markov-Chain Monte Carlo (MCMC) to approximate it.

There are however cases when the bayesian approach does work. Either the model
is small and simple enough so that the integral can be actually computed, or
its properties (such as conditional independence in LDA
\citep{lda-blei2003latent}) allow us to perform more efficient MCMC sampling,
or the model could actually have an analytic solution to the marginalization.
The last case is something we will make use of later on when we introduce
Gaussian Processes.

Unfortunately, in the field of deep learning \citep{dlbook} and neural
networks, our models are almost always so large that just computing a point
estimate of the parameters is close to our computational limits. For that
reason we do not usually employ bayesian methods, but look for alternative --
less computationally heavy -- approaches.

A simple approach to hyperparameter tuning is either via random search, where
each parameter is sampled randomly from a given prior distribution, or via grid
search, where a fixed grid over the hyperparameter space is chosen, and then
each point on the grid is evaluated, and the best parameter combination is
chosen. The evaluation can be done using an arbitrary metric of performance,
called the \newterm{objective function}. This could simply be the loss
achieved by the model on the validation set.

It is not difficult to see that neither of these approaches are optimal. Both
do not take into account the already computed values of the loss. If the model
is evaluated on one set of hyperparameters and achieves a high loss, we would
like the hyperparameter tuning mechanism to take this into account, and
possibly try a different combination. This process is similar to that of an
agent trying to balance the exploration-exploitation
tradeoff~\citep{russell2016artificial}.  On one hand, we would like to explore
different combinations of hyperparameters and explore as much of the search
space as possible.  But once we find a combination with a high value of the
objective function we would want to explore the space around that combination
to exploit the high value and possibly find a close combination that is even
better.

Therefore, we'd like our hyperparameter optimization procedure, also called the
\newterm{meta-optimizer} \todo{najit nejaky nazev a pouzivat konzistentne od
zacatku}, to balance the exploration-exploitation tradeoff, and take previous
evaluation into account when choosing which point to evaluate next. Since the
training of machine learning models -- and neural networks in particular -- can
be computationally costly, we need to pick an optimization procedure that is
sample efficient. Many modern deep learning models take on the order of days,
weeks, or even months to train on high end hardware. To give a few examples,
the recent OpenaI Five \citep{openai-five} has consumed $800$ PETAFLOPs-days
over the course of 10 months. In comparison, a consumer grade GPU GTX 1080 Ti
achieves just over 11 TFLOPs. A smaller and more realistic example would be the
NVIDIA StyleGAN \citep{nvidia-stylegan} trained on $1024\times1024$ images
takes almost $7$ days on $8\times$ Tesla V100 GPUs. In our experiments (see
\autoref{chapter:experiments}) we train a tagger and lemmatizer on Czech and
English treebanks, where each evaluation takes about four GPU hours. Evaluating
the objective function would mean training one such model from scratch with a
given set of hyperparameters to obtain an unbiased estimate of its performance.

Even though we have the ability to evaluate the objective function, we have no
way of computing its gradients, or obtain its analytic form. As a result we
have to treat it as a black box and use optimization techniques which don't
require either. But even with the smaller models we have just shown, it is easy
to see that we can not perform more than a few hundred of evaluations without
spending unreasonable costs on computational resources. This immediately
disqualifies many of the common black box optimization techniques, such as
evolution strategies or simmulated annealing, which require on the order of
thousands evaluations to converge \citep{google-vizier}.

Bayesian optimization \citep{nando-bayesian-out-of-the-loop} is a black box
optimization technique which utilizes a probabilistic model to take a set of
evaluations of the objective function and computes a posterior over all
possible functions give the observed data.  As the next step, it computes an
\newterm{acquisition function}, which is a function of the posterior, and
represents the potential usefulness of the next sample. A popular example is
the \newterm{expected improvement} function (\autoref{chapter-ei}), which is
roughly defined as \emph{the expected improvement over the maximum obtained so
far}. By sampling at the maximum of the acquisition function we obtain the
point that is most likely going to help us the most. A key insight here is that
the model is probabilistic. It does not simply fit a regression line through
the data points and find the maximum.  Instead, we fit a \newterm{Gaussian
Process} (see \autoref{chapter:gp}) which allows us to capture the uncertainty in the
predictions. This uncertainty is taken into account by the acquisition
function, and as a result it ends up balancing the exploration-exploitation
tradeoff by both taking into account the predicted value, and our uncertainty
in that value.

% Even though theoretically possible \ref{learning-to-learn},computing the
% gradient with respect to a learning rate is not done in
% TODO: https://openreview.net/pdf?id=BkrsAzWAb



% https://openai.com/blog/how-to-train-your-openai-five/


\section{Our Contributions}

We implemented a tool for optimizing arbitrary programs' hyperparameters using
bayesian optimization, including a scheduler which runs the evaluations on a
cluster, and a web interface visualizing the results. We do not provide any
theoretical extensions to bayesian optimization -- the benefit is only in
implementation. This work however also serves as a thorough introduction to the
theoretical background, specifically on gaussian processes (GP \todo{zkratku
zacinat na zacatku}). Understanding the theoretical aspects of GPs allows the
user to interpret the behavior of the optimizer, as well as better understand
why some result visualizations might look the way they do.

Our implementation of bayesian optimization utilizes the GPy library \citep{gpy2014}
which implements the basic GP regression model. We chose to use the library
mainly for the reasons of numerical stability. In theory, as we will show later
\autoref{chapter:gp}, impelmenting GP regression is simple for a small toy example.
But with realistic data it is easy to run into numerical issues, some of which
we will go over in the following \autoref{chapter:bo}. Apart from
numerically stable code, the GPy allows for more control over the
hyperparameters of the GP using constrained optimization.

Fitting a GP model and optimizing the acquisition function is just the
beginning. A non-trivial amount of the work is devoted to evaluating the
function in a flexible way. In our case, we expect the user to provide a script
which encapsulates the function, accepts its parameters in the form of command
line arguments, and prints the result of the function on its standard output.
This approach allows us to run the evaluations in parallel, or even run them on
a cluster. The implementation is flexible enough so that a user could provide
their own way of running the evaluations, should they have specific needs.  The
experimental data is also stored in an easy to access text format, with command
line utilities that allow the user a fine grained control over the experiment.

Lastly, running real-life experiments can be a time consuming process, and
having the ability to monitor the process and interfere if needed is an
important feature. This is why we provide a web interface that can visualize
the progress of the optimization. The user can look at the evaluated samples,
as well as the regression model at any point in time during the optimization.



% We extend it with an implementation of the acquisition functions \ref{chapter-acq}. 
